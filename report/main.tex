\documentclass{article}
\usepackage[utf8]{inputenc}
\usepackage[portuguese]{babel}
\usepackage{enumitem}
\usepackage{a4wide}
\usepackage[a4paper, margin=1in]{geometry}
\usepackage{graphicx}
\usepackage{xcolor}
\usepackage{url}
\usepackage{hyperref}
\usepackage[parfill]{parskip}
\usepackage{fancyhdr}

\def\BibTeX{{\rm B\kern-.05em{\sc i\kern-.025em b}\kern-.08em
    T\kern-.1667em\lower.7ex\hbox{E}\kern-.125emX}}
    
\pagestyle{fancy}
\fancyhf{}

\renewcommand{\headrulewidth}{0.03cm}% 2pt header rule
\renewcommand{\headrule}{\hbox to\headwidth{%
  \color{castanho-ue}\leaders\hrule height \headrulewidth\hfill}}
  
\renewcommand{\footrulewidth}{0.03cm}% 2pt header rule
\renewcommand{\footrule}{\hbox to\headwidth{%
  \color{castanho-ue}\leaders\hrule height \headrulewidth\hfill}}
  
\rfoot{\thepage}

\definecolor{castanho-ue}{HTML}{ad4758}

\pagenumbering{roman}
\begin{document}

\begin{titlepage}

\begin{minipage}{0.7\textwidth}
\noindent\LARGE\textbf{Front-end Developer\\\\}
\Large{Licenciatura em Eng. Informática\\}
\large{Estágio-Projeto 2021-2022\\}
\end{minipage}
\begin{minipage}[t]{0.3\textwidth}\raggedleft
\includegraphics[width=.9\linewidth]{images/di.pdf}
\end{minipage}
\noindent
\includegraphics[trim={0 2cm 0 0}, clip,  width=\linewidth]{images/claustros.png}\\

\vspace{0.5cm}
\noindent\textcolor{castanho-ue}{\rule{\textwidth}{0.03cm}}

\vspace{0.5cm}
\noindent\large\textbf{Ricardo Marques Oliveira\\}

\noindent
Orientador na empresa: Fábio Belga\\     
Orientador no departamento: Professor José Saias

\emph{\\Trabalho desenvolvido na empresa Optiply no âmbito da disciplina de Estágio-Projeto da Licenciatura em Eng. Informática.}\\ 

\begin{flushright}
\emph{Évora, \today}
\end{flushright}

\end{titlepage}


\cleardoublepage
\tableofcontents

\cleardoublepage
\pagenumbering{arabic}
\section{Introdução}
\hspace*{0.5cm} Este relatório foi elaborado no âmbito da licenciatura em Engenharia Informática, pertencente ao Departamento de Informática da Escola de Ciências e Tecnologia da Universidade de Évora, pela Unidade Curricular Estágio-Projeto, desempenhado na empresa Optiply. Tem como objetivo a apresentação e descrição de toda a atividade desempenhada, bem como das tecnologias e métodos aplicados no decorrer do estágio. \newline
\hspace*{0.5cm} A educação em contexto laboral, recentemente adotada para a licenciatura em Engenharia Informática, prepara os alunos e proporciona a experiência prática dos conceitos adquiridos no decorrer da formatura, em contexto real no mercado de trabalho propriamente dito. Promove assim, o desenvolvimento de competências, tanto técnicas como organizacionais, cruciais para o futuro. \newline
\hspace*{0.5cm} Em simultâneo, para a instituição que alberga os formandos também retira proveito, uma vez que, facilita os processos de recrutamento de novo pessoal, proporcionando assim também uma constante adaptação à evolução tão constante que existe na área da tecnologia. \newline
\hspace*{0.5cm} O estágio promoveu também, para auxílio mais ativo ao aluno, um orientador na empresa e um orientador no departamento.

\hspace*{0.5cm} As ferramentas utilizadas e mais presentes no decorrer deste período foram \emph{Angular} \cite{angular, angular-wiki, angular-docs, angular-repo} e por consequente, \emph{TypeScript} \cite{ts, ts-docs}, uma vez que \emph{Angular} é baseado nesta linguagem de programação; bem como, uma vez que o projeto baseava-se no desenvolvimento \emph{front-end} \cite{frontend}, \emph{HTML} \cite{html-article, html-wiki} e \emph{CSS} \cite{css}. \newline
\hspace*{0.5cm} Como ferramenta essencial, \emph{Angular}, esta plataforma foi desenvolvida pela \emph{Google} \cite{google} originalmente designada por \emph{AngularJS} \cite{old-angular} (para as versões 1.X da \emph{framework}), foi reescrita pela mesma equipa inicial, estando esta atualmente na versão \textbf{14.0.0}. \newline

\subsection{Enquadramento}
\hspace*{0.5cm} O estágio foi realizado no âmbito da Unidade Curricular Estágio-Projeto da licenciatura em Engenharia Informática da Universidade de Évora, como já foi mencionado, para a empresa Optiply. Empresa esta que é Holandesa, no entanto está sediada também em Portugal, na cidade de Évora. \newline 
\hspace*{0.5cm} Este foi realizado em regime misto, isto é, existiu a flexibilidade e autogestão no tipo de regime que pretendíamos optar. Ou seja, tanto poderíamos realizar uma semana presencialmente nos escritórios da sede da empresa, como realizar as tarefas remotamente. \newline
\hspace*{0.5cm} O mesmo teve duração de 240 horas, tendo início em fevereiro e término no final do mês de maio, sendo que foram realizadas 8 horas diárias, 2 vezes por semana, somando assim 16 horas semanais, ao longo de 15 semanas. Concretamente, teve início a 14 de fevereiro do corrente ano e término a 25 de maio do mesmo ano, concluindo assim a totalidade de horas determinadas. \newline

\subsection{Objetivos}
\hspace{0.5cm} De modo a demonstrar total perceção do que foi realizado no decorrer do estágio, nesta secção serão apresentados os objetivos, de forma integral, propriamente ditos.
\begin{itemize}
  \item Inicialmente, de modo a agilizar a aprendizagem da nova ferramenta \emph{Angular}, a empresa dispôs-nos um curso de introdução à \emph{framework}, através da plataforma de cursos \emph{Udemy} \cite{udemy}, com a duração de mais de 11 horas em aulas com formato de vídeo. Neste curso, a finalidade foi desenvolver 7 projetos distintos para uma perceção das diferentes ferramentas que a \emph{framework} oferece aos seus desenvolvedores;
  \item Após conclusão da etapa anteriormente descrita, iniciou-se a fase seguinte que consistiu na replicação da \emph{Web Application} \cite{webapp} já existente como serviço da empresa;
  \item Finalmente, integrando o último objetivo retratado, finalizou-se o desenvolvimento da mesma aplicação, mas desta vez, em formato \emph{Pair Programming} \cite{pp} em conjunto com o meu colega de curso e neste caso, também de estágio.
\end{itemize}

\vspace*{0.125cm}

\subsection{Contribuições}
\label{sec:cont}
\hspace*{0.5cm} Apesar de ter sido desenvolvida a aplicação quase na sua integridade, o seu caráter foi exclusivamente instrutivo para os formandos, uma vez que o serviço já foi desenvolvido por profissionais da empresa e continua em constante manutenção e atualização. No entanto, acredito que esta versão desenvolvida por nós pode vir a ser utilizada no futuro como plataforma de teste ou adaptada para outra vertente que surja à posteriori. \newline

\subsection{Estrutura do documento}
\hspace*{0.5cm} Após o ponto atual da introdução, seguir-se-á o ponto \hyperref[sec:amb-emp]{Ambiente empresarial}, onde irá ser descrita resumidamente a empresa e também em que envolvente fomos inseridos na mesma.

\hspace*{0.5cm} No \hyperref[sec:est-art]{ponto subsequente} ao último descrito, irão ser mencionadas as soluções propostas ao desafio encarado, bem como as ferramentas ou metodologias que foram adotadas aquando do decorrer das tarefas propostas no estágio.

\hspace*{0.5cm} Após o Estado da arte, prolongar-se-á para o ponto \hyperref[sec:amb-dev]{Ambiente de desenvolvimento}. É neste ponto que serão dadas algumas opiniões pessoais sobre a experiência vivenciada na empresa do ponto de vista de trabalho; é também onde se irão abordar tópicos mais técnicos relacionados com o \emph{hardware} e \emph{software} utilizado; bem como a metodologia de trabalho que a empresa nos proporcionou ou previu que exercêssemos.

\hspace{0.5cm} Segue-se após, a secção do \hyperref[sec:trab-dev]{Trabalho desenvolvido}, no qual será descrito detalhada e pormenorizadamente todo o trabalho realizado durante o período de estágio.

\hspace{0.5cm} Finalmente, previamente às \hyperref[referencias]{Referências} (que será a última página do documento), irá tomar lugar a \hyperref[sec:ava-crt]{Avaliação crítica}, secção final onde serão retratadas as últimas apreciações relativamente ao trabalho realizado.

\cleardoublepage
\section{Ambiente empresarial}
\label{sec:amb-emp}
\hspace*{0.5cm} A Optiply é uma empresa que foi fundada em 2015 em Amesterdão, Holanda por Wiebe Konter; neste momento conta com mais de 50 profissionais, em pelo menos 4 países. A empresa encontra-se numa fase de evolução e esperam começar a operar em mais países da Europa e, se possível, até ao final do ano, ingressar no mercado Asiático. A empresa baseia-se num serviço prestado a terceiros de eficiência de \emph{Supply Chain} \cite{supp-chain}, isto é, uma outra empresa que contrata os serviços da Optiply, poderá ver um aumento no seu volume de negócio, menor \emph{stock} "bloqueado" em armazém e menor tempo desperdiçado em contacto com os fornecedores, uma vez que este serviço automatiza todo o processo e efetua a previsão de vendas para cada produto, delicadamente para cada empresa e cada tipo de mercado em que estas operam. \newline
\hspace*{0.5cm} Os diversos profissionais pertencentes à empresa, operam em distintos setores, tais como, os diferentes ramos das tecnologias de informação, recursos humanos, gestão, consultoria, entre outros. Dos quais tivemos oportunidade de contactar diretamente em contexto profissional. A equipa é constituída na sua grande maioria por novos talentos jovens e proporciona bastantes estágios curriculares nas diversas áreas em que opera, nos diferentes países. \textbf{TODO: EXPLICAR COMO SE "DIVIDEM" AS EQUIPAS NA EMPRESA}\newline
\hspace*{0.5cm} A sede da Optiply em Portugal, como já foi referido, encontra-se na cidade de Évora e é constituída por 3 escritórios modernos e amplos; dos quais podemos contar com salas de reuniões, espaços de trabalho comuns e privados (de modo a permitir melhor comunicação nos trabalhos em equipa, como os que exigem maior concentração e necessitam de mais privacidade) e uma ampla sala comum (designada de \emph{lounge}), onde os funcionários podem fazer uma pausa. A empresa consegue assim, promover um espírito de equipa e cooperação bastante assente, bem como um excelente ambiente entre todos os trabalhadores, não só a nível nacional, mas também com os colaboradores de outros países. O ambiente que se instala nas instalações, é o equilíbrio perfeito entre o formal/informal e profissional/social. Com isto, todos os estagiários que deram entrada, em simultâneo, na empresa no passado mês de fevereiro sentiram-se bastante bem acolhidos e integrados nos seus postos. \newline
\hspace*{0.5cm} A empresa acolheu 4 alunos de 2 instituições distintas, a Universidade de Évora e o Instituto Politécnico de Beja; criando assim a possibilidade de contacto entre pessoas com métodos de formação totalmente distintos, podendo assim abranger alguns conhecimentos e metodologias. Fui integrado na equipa de \emph{Front-end developers}, no entanto, como mencionado nas \hyperref[sec:cont]{Contribuições}, não lidei diretamente com o cliente. Esta mesma equipa, liderada pelo \emph{Development Team Lead} e também Orientador na empresa, Fábio Belga, é constituída na sua totalidade por colaboradores portugueses, com os quais consegui ter contacto direto quando realizei as tarefas em regime presencial. \newline

\cleardoublepage
\section{Estado da arte}
\label{sec:est-art}
Se no âmbito do seu trabalho de estágio fez o levantamento e/ou análise de várias soluções, ferramentas ou metodologias, com o objetivo de escolher a melhor, deve incluir uma descrição desse trabalho nesta secção, incluindo as conclusões a que chegou.
Esta secção pode não fazer sentido em alguns trabalhos.


\cleardoublepage
\section{Ambiente de desenvolvimento}
\label{sec:amb-dev}
\hspace*{0.5cm} Ao iniciar o estágio, realizou-se uma breve reunião de boas-vindas, na qual estiveram presentes, para além dos novos estagiários, o \emph{Development Team Lead} - Fábio Belga e o \emph{Talent Acquisition} - João Fernandes (com o qual realizámos toda a fase de recrutamento). Nesta conferência, foi-nos apresentada novamente a empresa, desta vez de modo mais detalhado, explicado quais seriam as tarefas para ambas as equipas (quer de \emph{Front-end Developers}, como \emph{Back-end Developers}). Nesta etapa foi-nos também atribuído um \emph{e-mail} da empresa, o qual deveríamos começar a utilizar para qualquer situação relacionada com o estágio. \newline

\subsection{Ambiente técnico}
\subsubsection{Hardware}
\hspace*{0.5cm} A empresa não concedeu qualquer tipo de máquina aos alunos que realizaram o estágio, no entanto, sempre que nos deslocámos à empresa, tivemos à disposição todo o tipo de periféricos para proporcionar uma melhor prestação e maior conforto. \newline

\hspace*{0.5cm} Assim sendo, utilizei a minha máquina pessoal para realizar o estágio, sendo ela constituída por um processador \emph{Intel Core i7 9750H}, 16GB de memória, \emph{NVIDIA GeForce RTX 2060} como placa gráfica e 2 discos rígidos, de 512GB SSD NVMe e outro de 1TB SSD M.2. Possui um \emph{dual-boot} com opção entre os 2 Sistemas Operativos que utilizo, sendo eles Linux com distribuição Ubuntu na versão 21.10 e Windows 10 Pro; de realçar que durante todas as tarefas do estágio foi utilizado exclusivamente Ubuntu. \newline

\hspace*{0.5cm} Aos servidores com os quais a empresa atua, não nos foi garantido acesso, uma vez que não seria necessário para a realização de nenhuma tarefa designada durante o decorrer do estágio. \newline

\subsubsection{Software}
\hspace*{0.5cm} Em contexto empresarial/profissional, na área das tecnologias de informação, um \emph{software} essencial na comunicação entre todos os envolventes de uma empresa é o \emph{Slack} \cite{slack}, plataforma que nos foi logo apresentada e demonstrada. Através desta, existiu o contacto tanto com membros internos como externos à equipa em que estava incluído. \newline
\hspace*{0.5cm} Para a realização propriamente dita das tarefas atribuídas no decorrer do período do estágio, foram utilizadas diversas ferramentas que são bastante utilizadas atualmente pela maioria das empresas da área, nomeadamente:
\begin{itemize}
    \item \textbf{Angular} - plataforma de desenvolvimento de aplicações \emph{web} e \emph{front-end} utilizada como base para a realização de todas as tarefas; uma vez que é desenvolvida sobre \emph{TypeScript}, este \emph{software} acabou por ser também utilizado. Associado também ao \emph{Angular}, foi utilizada a biblioteca (\emph{Librabry}) da mesma, \emph{Angular Material}. Esta biblioteca contem \emph{user interfaces} por defeito que podem ser utilizadas, de modo a acelerar o processo de desenvolvimento, evitando assim que o desenvolvedor crie constantemente o mesmo tipo de elementos. Foi a primeira vez que tive interação com esta ferramenta, de modo que não foi acessível numa fase inicial, porém, com o decorrer das atividades, habituei-me rapidamente.
    \item \textbf{TypeScript} - linguagem de programação que teve que ser inevitavelmente utilizada, uma vez que \emph{Angular} é baseado nela. Mesmo sendo a primeira vez que tive contacto com esta linguagem, sendo ela um \emph{superset} de \emph{JavaScript} (linguagem esta com a qual já estava familiarizado, uma vez que foi alvo de estudo em diversas Unidades Curriculares da licenciatura), adaptei-me com bastante facilidade às características da mesma.
    \item \textbf{HTML} - linguagem utilizada de forma consideravelmente reduzida e igualmente, como objeto de estudo em contexto académico, já tinha tido contacto com esta, logo não enfrentei qualquer dificuldade na utilização da mesma.
    \item \textbf{CSS} - este mecanismo de estilização é aplicado diretamente num documento \emph{web}, à semelhança de \emph{HTML}, também já tinha tido contacto com esta ferramenta, no entanto adquiri novos conceitos da mesma.
    \item \textbf{Jira} - ferramenta que permite a monitorização e acompanhamento das tarefas de um projeto em tempo real por parte de uma equipa de trabalho. Ferramenta esta que é bastante utilizada em contexto real de trabalho e a qual nunca tinha tido contacto, no entanto é bastante percetível e de fácil manuseamento.
    \item \textbf{Bitbucket} - sistema de controlo de versões distribuído e hospedeamento de projetos. Ferramenta que, à semelhança do \emph{Jira}, é bastante útil profissionalmente e nunca tinha tido também interação com a mesma.
    \item \textbf{Github} - também é uma plataforma de sistema de controlo de versões, utilizada anteriormente por mim e utilizada nas tarefas feitas de forma singular.
\end{itemize}

\subsubsection{Ambiente aplicacional}
\hspace*{0.5cm} O projeto desenvolvido, no que toca ao seu contexto aplicacional, uma vez que, como já referido, foi criar uma réplica do \emph{front-end} da aplicação do serviço da empresa, considera-se que este é \emph{standalone} não interagindo assim com outros sistemas da empresa; correndo assim, somente localmente na máquina que utilizei no decorrer do estágio. Podemos considerar que a única interação que esta teve, foi com \emph{RESTful APIs} \cite{rest-api} (\emph{Representational State Transfer Application Programming Interface}) de um modo genérico para utilizar dados na criação de elementos do \emph{Angular Material} que fossem preenchidos automaticamente. \newline

\subsection{Metodologia de trabalho}
\hspace*{0.5cm} Num fase inicial do estágio, de modo a uma mais rápida introdução e aprendizagem da \emph{framework Angular}, fomos introduzidos a um curso da mesma, através da plataforma de \emph{e-learning Udemy}. O curso propriamente dito tinha a duração de mais de 11 horas com formato de vídeo, com a finalidade de termos contacto com as principais mecanismos da \emph{framework}, de modo a conseguirmos iniciar o projeto propriamente dito. No entanto, como o curso abrangia um total de 7 pequenos projetos, a duração do mesmo prologou-se para mais horas, devido ao tempo despendido na realização desses mesmos projetos. \newline
\hspace*{0.5cm} Na fase seguinte, quando abordámos o projeto pela primeira vez, foi feito de forma singular, isto é, não foi realizado trabalho em equipa/grupo, logo a gestão e planificação do mesmo foi feita de forma livre por cada um dos formandos. Tomei então a decisão de utilizar o \emph{Github} como controlador de versões e o \emph{Trello} para planificar as minhas tarefas (plataforma idêntica ao \emph{Jira}, mas esta ainda não me tinha sido apresentada). \newline
\hspace*{0.5cm} Na última fase do projeto, trabalhei em equipa, utilizando desta vez o \emph{Jira} com um único \emph{board} dividido em 3 colunas (\emph{TO DO}, \emph{IN PROGRESS} e \emph{DONE}), juntamente com o \emph{Backlog} do mesmo (não utilizando a opção de \emph{sprints} propriamente dita). Assim, adaptou-se um desenvolvimento ágil de \emph{software}, \emph{Pair Programming} mais propriamente dito. \newline
\hspace*{0.5cm} Semanalmente reunimos com o \emph{Development Team Lead} de modo a discutirmos o trabalho que estava a ser realizado, tirar alguma questão mais complicada (que não tivéssemos oportunidade de apresentar aos restantes membros da equipa de \emph{Front-end developers}) e/ou serem apresentadas novas tarefas. \newline

\cleardoublepage
\section{Trabalho desenvolvido}
\label{sec:trab-dev}
Esta secção deve começar com um resumo claro e bem estruturado. Deve apresentar de forma resumida o trabalho realizado durante o seu estágio. Este resumo deve incluir a lista de tarefas realizadas e timeline respectivo.

Exemplo de um possível resumo:

Quando entrei para a empresa, fui enviado para um curso de formação de quatro semanas para aprender Ruby e como usar as várias ferramentas utilizadas regularmente no departamento. O curso também me apresentou os padrões e procedimentos da empresa.

Após a conclusão do curso, juntei-me à equipa de cinco pessoas responsáveis por manter o sistema de contabilidade de compras. A principal tarefa da equipa naquele momento era modificar o sistema para que os gerentes pudessem fazer consultas on-line sobre ordens de compra pendentes. Recebi dois pequenos programas para escrever e realizar testes unitários. Já existia um projeto detalhado para cada um dos programas: cada um tinha cerca de 200 instruções; escrever e testar os dois demorou cerca de oito semanas.

Fui então convidado a ajudar na produção dos dados de teste para a versão on-line do sistema. Isso envolveu uma leitura cuidadosa da especificação funcional, o que me levou a perceber melhor o que era feito pelo sistema. Esta tarefa demorou cerca de dez semanas para produzir os dados necessários. Os dados de teste foram revistos por outros membros da equipa e levámos mais duas semanas para incorporar as modificações solicitadas.

\subsection{Descrição detalhada}
Nesta secção deve  selecionar e ampliar, em sub-secções separadas, as partes do trabalho desenvolvido no âmbito do seu estágio que lhe parecem mais relevantes e interessantes.

\cleardoublepage
\section{Avaliação crítica}
\label{sec:ava-crt}
Nesta secção deve apresentar o que aprendeu com a experiência, evitando ser trivial. Além de discutir o que aprendeu do ponto de vista técnico, deve escrever algo sobre a sua experiência não técnica, em áreas como o trabalho em equipa e a comunicação.

Alguns exemplos de comentários que pode escrever nesta secção:

"Fiquei surpreendido com a extensão dos testes realizados. Eu sabia que os testes eram importantes, mas não tinha percebido que isso significava seguir a especificação funcional e testar cada instrução implementada. Demorou mais tempo para testar um módulo do que  implementá-lo.

Dada a seriedade com que os testes foram realizados, foi estranho descobrir que outras formas de gestão de qualidade eram quase inexistentes. Não existiu uma revisão formal das especificações e não houve revisão do código. Não havia gestão de configuração e ocorreram vários erros porque foram usadas versões erradas de alguns módulos.

Tecnicamente, acho que não aprendi muito com o meu estágio, excepto como programar em Ruby - uma experiência que espero nunca mais repetir. Porém, aprendi muito sobre a forma como as equipas de desenvolvimento de software trabalham e o que significa ter utilizadores reais."


\cleardoublepage
\bibliographystyle{unsrt}
\bibliography{bibs}
\label{referencias}

\end{document}
